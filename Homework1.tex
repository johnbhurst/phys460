\documentclass[12pt]{extarticle}
\usepackage{phys460}

\title{PHYS460 - Homework 1}
\author{John Hurst}
\date{July 2024}

\hyphenation{trans-pi-la-tion}
\begin{document}
\maketitle

%%%%%%%%%%%%%%%%%%%%%%%%%%%%%%%%%%%%%%%%%%%%%%%%%%%%%%%%%%%%%%%%%%%%%%%%%%%%%%%%%%%%%%%%%%%%%%%%%%%%
We have discussed the circuit representation of quantum operations $\calE(\rho)$ acting on system states $\rho$.
See Figure~\ref{fig:system_in_environment}.
The essence was that a quantum operation has the general form
\begin{equation}\label{eq:quantum_operation}
\calE(\rho) = \sum_k E_k \rho E_k^\dagger,
\end{equation}
with operation elements $E_k$ related to the unitary transformation $U$ via $E_k = \matrixelement{e_k}{U}{e_0}$.
Here the $\ket{e_k}$ form an orthonormal basis in the environment's ket space,
which needs to be large enough to accommodate all of the $E_k$.
Note that the latter are not numbers,
but operators acting in the system Hilbert space!
The output state of the combined system + environment after $U$ is then
\begin{equation}\label{eq:output_state}
U[\rho\otimes\outerproduct{e_0}{e_0}]U^\dagger = \sum_{kk'} E_k \rho E_{k'}^\dagger \otimes \outerproduct{e_k}{e_{k'}},
\end{equation}
and a measurement of the environment in the $\{\ket{e_k}\}$ basis results in the system to be in the state $\calE(\rho)$ given in Eq.~\eqref{eq:quantum_operation}.

\begin{figure}[h]
\centering
$\begin{array}{c}
\Qcircuit @C=.5em @R=.8em {
\lstick{\rho} & \multigate{1}{U} & \rstick{\calE(\rho)} \qw  \\
\lstick{\ket{e_0}} & \ghost{U} & \qw & \meter \qw
}
\end{array}$
\caption{}
\label{fig:system_in_environment}
\end{figure}

%%%%%%%%%%%%%%%%%%%%%%%%%%%%%%%%%%%%%%%%%%%%%%%%%%%%%%%%%%%%%%%%%%%%%%%%%%%%%%%%%%%%%%%%%%%%%%%%%%%%
\question{Example 1}{Consider the quantum-operation circuit shown in Figure~\ref{fig:system_Ry} below.

\begin{figure}[h]
\centering
$\begin{array}{c}
\Qcircuit @C=.5em @R=.8em {
\lstick{\rho} & \ctrl{1} & \rstick{\calE(\rho)} \qw  \\
\lstick{\ket{e_0}} & \gate{R_y(\theta)} & \qw & \meter \qw
}
\end{array}$
\caption{}
\label{fig:system_Ry}
\end{figure}

Here both system and environment are a single qubit.
Find the unitary matrix $U$ and, from that, determine the operation elements $E_0$ and $E_1$ so that you have
$\calE(\rho) = E_0 \rho {E_0}^\dagger + E_1 \rho {E_1}^\dagger$.
Apply the quantum operation $\calE(\rho)$ to the general form of the system density matrix,
\begin{equation}\label{eq:rho_general}
    \rho = \begin{pmatrix}a & b \\ c & d\end{pmatrix}.
\end{equation}
Describe the effect that the operation has on the system state.
}

The unitary matrix $U$ for this example is given by
\begin{align*}
U & = \outerproduct{0}{0}\otimes I + \outerproduct{1}{1}\otimes R_y(\theta) \\
& = \begin{pmatrix}
    1 & 0 & 0 & 0 \\
    0 & 1 & 0 & 0 \\
    0 & 0 & \cos\frac{\theta}{2} & -\sin\frac{\theta}{2} \\
    0 & 0 & \sin\frac{\theta}{2} & \cos\frac{\theta}{2}
\end{pmatrix}.
\end{align*}

To determine the operation elements (Kraus operators) $E_0$ and $E_1$,
I'll follow Nielsen and Chuang\cite{nielsen2016} page p.363,
and use their notation, specifically using $\ket{0_P},\ket{1_P}$ for the principal system basis, and $\ket{0_E},\ket{1_E}$ for the environment basis.
\begin{align*}
U & = \outerproduct{0_P}{0_P} \otimes I + \outerproduct{1_P}{1_P} \otimes R_y(\theta) \\
& = \outerproduct{0_P}{0_P} \otimes \outerproduct{0_E}{0_E} + \outerproduct{0_P}{0_P} \otimes \outerproduct{1_E}{1_E} \\
& \qquad + \outerproduct{1_P}{1_P} \left[ \cos\frac{\theta}{2} \outerproduct{0_E}{0_E} - \sin\frac{\theta}{2} \outerproduct{0_E}{1_E} + \sin\frac{\theta}{2} \outerproduct{1_E}{0_E} + \cos\frac{\theta}{2} \outerproduct{1_E}{1_E} \right] \\
E_0 & = \mel{0_E}{U}{0_E} = \outerproduct{0_P}{0_P} + \cos\frac{\theta}{2} \outerproduct{1_P}{1_P}
= \begin{pmatrix}1 & 0 \\ 0 & \cos\frac{\theta}{2}\end{pmatrix}, \\
E_1 & = \mel{1_E}{U}{0_E} = \sin\frac{\theta}{2} \outerproduct{1_P}{1_P}
= \begin{pmatrix}0 & 0 \\ 0 & \sin\frac{\theta}{2}\end{pmatrix}.
\end{align*}
These matrices correspond to the phase damping example in Nielsen and Chuang page 384:
\begin{align*}
E_0 & = \begin{pmatrix}1 & 0 \\ 0 & \sqrt{1-\lambda}\end{pmatrix}, \\
E_1 & = \begin{pmatrix}0 & 0 \\ 0 & \sqrt{\lambda}\end{pmatrix},
\end{align*}
where in their example $\lambda = 1 - \cos^2(\chi\Delta t)$ ``can be interpreted as the probability that a photon from the system has been scattered (without loss of energy).''

Applying to the general form of the system density matrix:
\begin{align*}
\calE(\rho) & = E_0 \rho {E_0}^\dagger + E_1 \rho {E_1}^\dagger \\
& = \begin{pmatrix}1 & 0 \\ 0 & \cos\frac{\theta}{2}\end{pmatrix} \begin{pmatrix}a & b \\ c & d\end{pmatrix} \begin{pmatrix}1 & 0 \\ 0 & \cos\frac{\theta}{2}\end{pmatrix}
+ \begin{pmatrix}0 & 0 \\ 0 & \sin\frac{\theta}{2}\end{pmatrix} \begin{pmatrix}a & b \\ c & d\end{pmatrix} \begin{pmatrix}0 & 0 \\ 0 & \sin\frac{\theta}{2}\end{pmatrix} \\
& = \begin{pmatrix}a & b\cos\frac{\theta}{2} \\ c\cos\frac{\theta}{2} & d\cos^2\frac{\theta}{2}\end{pmatrix} + \begin{pmatrix}0 & 0 \\ 0 & d\sin^2\frac{\theta}{2}\end{pmatrix} \\
& = \begin{pmatrix}a & b\cos\frac{\theta}{2} \\ c\cos\frac{\theta}{2} & d\end{pmatrix}.
\end{align*}
This result agrees with the formula given in Lee, Ji \& Chen\cite{lee2024} for general phase damping:
\[
\calN_{\text{dephase}}(\rho) = (1-s)\rho + s\text{diag}(\rho),
\]
where $s$ is the probability of noise and $\text{diag}(\rho)$ is the diagonal part of $\rho$.

The dephasing process suppresses the off-diagonal elements of the density matrix, but does not effect the diagonal elements.

As described in Nielsen and Chuang, the effect of this operation is to project a Bloch vector along the $z$ axis,
losing the $x$ and $y$ components of the vector.

Physically, the relative phase between the basis states $\ket{0}$ and $\ket{1}$ becomes randomised.

%%%%%%%%%%%%%%%%%%%%%%%%%%%%%%%%%%%%%%%%%%%%%%%%%%%%%%%%%%%%%%%%%%%%%%%%%%%%%%%%%%%%%%%%%%%%%%%%%%%%
\question{Example 2}{Now consider the quantum-operation circuit depicted in Figure~\ref{fig:system_rho_Z}.

\begin{figure}[h]
\centering
$\begin{array}{c}
\Qcircuit @C=.5em @R=.8em {
\lstick{\rho} & \gate{Z} & \rstick{\calE(\rho)} \qw  \\
\lstick{\begin{pmatrix}p & 0 \\ 0 & 1-p\end{pmatrix}} & \ctrl{-1} & \qw & \meter \qw
}
\end{array}$
\caption{}
\label{fig:system_rho_Z}
\end{figure}

Here the environment qubit starts out in a mixed state, but the situation is still straighforward to analyse.
Find $\calE(\rho)$ and determine the effect of this operation on the general system state from Eq.~\eqref{eq:rho_general}.
Compare with the operation considered in Example 1 above.
}

In this situation the environment is an ensemble of qubit states where a given qubit has a $p$ probability of being in the $\ket{0}$ state and a $1-p$ probability of being in the $\ket{1}$ state.

When the environment qubit is in the $\ket{0}$ state, the principal system is unchanged.
When the environment qubit is in the $\ket{1}$ state, the principal system has the $Z$ operation applied.

We can describe this using the evolution of the density matrix:
\begin{align*}
\calE(\rho) & = p I \rho I + (1-p) Z \rho Z \\
& = p \begin{pmatrix}a & b \\ c & d\end{pmatrix} + (1-p) \begin{pmatrix}a & -b \\ -c & d\end{pmatrix} \\
& = \begin{pmatrix}a & (2p-1)b \\ (2p-1)c & d\end{pmatrix} \\
\end{align*}

This result is the same as in the previous question, with $\cos\frac{\theta}{2} = 2p-1$.

\question{Physical interpretation}{The quantum operations considered in Examples 1 and 2 above represent phase damping,
which is a particular noise source for qubits.
Briefly discuss its specifics.
}

According to Nielsen and Chuang, phase damping is a noise process that describes the loss of quantum information without loss of energy.
They give the example of a photon scattering randomly as it travels through a waveguide,
and another example of electronic states in an atom are perturbed upon interacting with distant electrical charges.

Most interesting is the observation that continuous phase errors are equivalent to discrete phase flips, making practical quantum error correction possible (p. 385):

\begin{displayquote}
    Historically, phase damping was a process that was almost always thought of, physically,
    as resulting from a random phase kick or scattering process. It was not until the
    connection to the phase flip channel was discovered that quantum error-correction was
    developed, since it was thought that phase errors were continuous and couldn't be described
    as a discrete process! In fact, single qubit phase errors can always be thought of
    as resulting from a process in which either nothing happens to a qubit, with probability
    $\alpha$, or with probability $1-\alpha$, the qubit is flipped by the Z Pauli operation. Although this
    might not be the actual microscopic physical process happening, from the standpoint of
    the transformation occurring to a qubit over a discrete time interval large compared to
    the underlying random process, there is no difference at all.
\end{displayquote}


\printbibliography
\addcontentsline{toc}{section}{References}

\end{document}
