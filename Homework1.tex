\documentclass[12pt]{extarticle}
\usepackage{phys460}

\title{PHYS460 - Homework 1}
\author{John Hurst}
\date{July 2024}

\hyphenation{trans-pi-la-tion}
\begin{document}
\maketitle

%%%%%%%%%%%%%%%%%%%%%%%%%%%%%%%%%%%%%%%%%%%%%%%%%%%%%%%%%%%%%%%%%%%%%%%%%%%%%%%%%%%%%%%%%%%%%%%%%%%%
We have discussed the circuit representation of quantum operations $\calE(\rho)$ acting on system states $\rho$.
See Figure~\ref{fig:system_in_environment}.
The essence was that a quantum operation has the general form
\begin{equation}\label{eq:quantum_operation}
\calE(\rho) = \sum_k E_k \rho E_k^\dagger,
\end{equation}
with operation elements $E_k$ related to the unitary transformation $U$ via $E_k = \matrixelement{e_k}{U}{e_0}$.
Here the $\ket{E-k}$ form an orthonormal basis in the environment's ket space,
which needs to be large enough to accommodate all of the $E_k$.
Note that the latter are not numbers,
but operators acting in the system Hilbert space!
The output state of the combined system + environment after $U$ is then
\begin{equation}\label{eq:output_state}
U[\rho\otimes\outerproduct{e_0}{e_0}]U^\dagger = \sum_{kk'} E_k \rho E_{k'}^\dagger \otimes \outerproduct{e_k}{e_{k'}},
\end{equation}
and a measurement of the environment in the $\{\ket{e_k}\}$ basis results in the system to be in the state $\calE(\rho)$ given in Eq.~\eqref{eq:quantum_operation}.

\begin{figure}[h]
\centering
$\begin{array}{c}
\Qcircuit @C=.5em @R=.8em {
\lstick{\rho} & \multigate{1}{U} & \rstick{\calE(\rho)} \qw  \\
\lstick{\ket{e_0}} & \ghost{U} & \qw & \meter \qw
}
\end{array}$
\caption{}
\label{fig:system_in_environment}
\end{figure}

\textbf{Example 1.} Consider the quantum-operation circuit shown in Figure~\ref{fig:system_Ry} below.

\begin{figure}[h]
\centering
$\begin{array}{c}
\Qcircuit @C=.5em @R=.8em {
\lstick{\rho} & \ctrl{1} & \rstick{\calE(\rho)} \qw  \\
\lstick{\ket{e_0}} & \gate{R_y(\theta)} & \qw & \meter \qw
}
\end{array}$
\caption{}
\label{fig:system_Ry}
\end{figure}

Here both system and environment are a single qubit.
Find the unitary matrix $U$ and, from that, determine the operation elements $E_0$ and $E_1$ so that you have
$\calE(\rho) = E_0 \rho E_0^\dagger + E_1 \rho E_1^\dagger$.
Apply the quantum operation $\calE(\rho)$ to the general form of the system density matrix,
\begin{equation}\label{eq:rho_general}
    \rho = \begin{pmatrix}a & b \\ c & d\end{pmatrix}.
\end{equation}
Describe the effect that the operation has on the system state.

\textbf{Example 2.} Now consider the quantum-operation circuit depicted in Figure~\ref{fig:system_rho_Z}.

\begin{figure}[h]
\centering
$\begin{array}{c}
\Qcircuit @C=.5em @R=.8em {
\lstick{\rho} & \gate{Z} & \rstick{\calE(\rho)} \qw  \\
\lstick{\begin{pmatrix}p & 0 \\ 0 & 1-p\end{pmatrix}} & \ctrl{-1} & \qw & \meter \qw
}
\end{array}$
\caption{}
\label{fig:system_rho_Z}
\end{figure}

Here the environment qubit starts out in a mixed state, but the situation is still straighforward to anslyse.
Find $\calE(\rho)$ and determine the effect of this operation on the general system state from Eq.~\eqref{eq:rho_general}.
Compare with the operation considered in Example 1 above.

\textbf{Physical interpretation.} The quantum operations considered in Examples 1 and 2 above represent phase damping,
which is a particular noise source for qubits.
Briefly discuss its specifics.

% \printbibliography
% \addcontentsline{toc}{section}{References}

\end{document}
