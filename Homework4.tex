\documentclass[12pt]{extarticle}
\usepackage{phys460}

\title{PHYS460 - Homework 4}
\author{John Hurst}
\date{October 2024}

\begin{document}
\maketitle

%%%%%%%%%%%%%%%%%%%%%%%%%%%%%%%%%%%%%%%%%%%%%%%%%%%%%%%%%%%%%%%%%%%%%%%%%%%%%%%%%%%%%%%%%%%%%%%%%%%%
\question{A2.2}{Prove Lagrange's theorem: if $H$ is a subgroup of a finite group $G$, then the order of $H$ divides the order of $G$.
}

Let $m=|H|$, so that $H = \{h_1, h_2, \ldots, h_m\}$.

Choose $g_1 \in G \setminus H$, and define
\[
G_1 = \{g_1h_1, g_1h_2, \ldots, g_1h_m\}.
\]
Then:
\begin{itemize}
\item $G_1 \cap H = \emptyset$, for if $g_1h_i = h_j$ for some $i$ and $j$, then $h_jh_i^{-1}= g_1 \in H$.
\item The elements of $G_1$ are distinct, for if $g_1h_i = g_1h_j$ for some $i$ and $j$, then $\inv{g_1}g_1h_i = h_i = \inv{g_1}g_1h_j = h_j$.
\item Therefore, $|G_1| = m$.
\end{itemize}

If $G\setminus G_1$ is not empty, choose $g_2 \in G\setminus G_1 \setminus H$ and define $G_2$ again as above.

Continue this process until $G \setminus (G_1 \cup G_2 \cup G_3 \cup \ldots \cup G_k) = H$, and there are no more elements to choose.
Then
\begin{align*}
G & = G_1 \cup G_2 \cup \ldots \cup G_k \cup H \\
|G| & = |G_1| + |G_2| + \ldots + |G_k| + m \\
& = (k+1)m \\
& = (k+1)|H|.
\end{align*}

%%%%%%%%%%%%%%%%%%%%%%%%%%%%%%%%%%%%%%%%%%%%%%%%%%%%%%%%%%%%%%%%%%%%%%%%%%%%%%%%%%%%%%%%%%%%%%%%%%%%
\question{A2.13}{Show that every irreducible Abelian matrix group is one-dimensional.
}

%%%%%%%%%%%%%%%%%%%%%%%%%%%%%%%%%%%%%%%%%%%%%%%%%%%%%%%%%%%%%%%%%%%%%%%%%%%%%%%%%%%%%%%%%%%%%%%%%%%%
\question{A2.16}{$S_3$ is the group of permutations of three elements.
Support we order these as mapping 123 to 123; 231; 312; 213; 132, and 321, respectively.
Show that there exist two one-dimensional irreducible representations of $S_3$, one of which is trivial,
and the other of which is 1, 1, 1, -1, -1, -1, corresponding in order to the six permutations given earlier.
Also show that there exists a two-dimensional irreducible representation, with the matrices
\begin{align*}
\begin{pmatrix}
    1 & 0 \\
    0 & 1
\end{pmatrix}, &&
\frac{1}{2}
\begin{pmatrix}
    -1 & -\sqrt{3} \\
    \sqrt{3} & -1
\end{pmatrix}, &&
\frac{1}{2}
\begin{pmatrix}
    -1 & \sqrt{3} \\
    -\sqrt{3} & -1
\end{pmatrix}, \\
\begin{pmatrix}
    -1 & 0 \\
    0 & 1
\end{pmatrix}, &&
\frac{1}{2}
\begin{pmatrix}
    1 & \sqrt{3} \\
    \sqrt{3} & -1
\end{pmatrix}, &&
\frac{1}{2}
\begin{pmatrix}
    1 & -\sqrt{3} \\
    -\sqrt{3} & -1
\end{pmatrix}.
\end{align*}
Verify that the representations are orthogonal.
}

%%%%%%%%%%%%%%%%%%%%%%%%%%%%%%%%%%%%%%%%%%%%%%%%%%%%%%%%%%%%%%%%%%%%%%%%%%%%%%%%%%%%%%%%%%%%%%%%%%%%
\question{A2.24}{Using the results of exercise A2.16, construct the Fourier transform over $S_3$ and express it as a 6x6 unitary matrix.}

%%%%%%%%%%%%%%%%%%%%%%%%%%%%%%%%%%%%%%%%%%%%%%%%%%%%%%%%%%%%%%%%%%%%%%%%%%%%%%%%%%%%%%%%%%%%%%%%%%%%
\question{10.46}{Show that the stabilizer for the three qubit phase flip code is generated by $X_1X_2$ and $X_2X_3$.}

The subgroup $S\subset G_3$ given by $S = \{I, X_1X_2, X_2X_3, X_3X_1\}$ is generated by $X_1X_2$ and $X_2X_3$,
because $X_1, X_2, X_3$ commute, and:
\begin{align*}
X_1X_2 \cdot X_2X_3 & = X_1X_3 \\
X_1X_2 \cdot X_1X_2 & = X_2X_3 \cdot X_2X_3 = I \\
\end{align*}

The subspace fixed by $X_1X_2$ is spanned by these vectors:
\begin{align*}
u_1 & = \ket{000} + \ket{110} \\
u_2 & = \ket{001} + \ket{111} \\
u_3 & = \ket{010} + \ket{100} \\
u_4 & = \ket{011} + \ket{101} \\
\end{align*}

The subspace fixed by $X_2X_3$ is spanned by these vectors:
\begin{align*}
v_1 & = \ket{000} + \ket{011} \\
v_2 & = \ket{001} + \ket{010} \\
v_3 & = \ket{100} + \ket{111} \\
v_4 & = \ket{101} + \ket{110} \\
\end{align*}

For a vector $w$ in the intersection of these subspaces, we have:
\begin{align*}
w & = a u_1 + b u_2 + c u_3 + d u_4 \\
  & = e v_1 + f v_2 + g v_3 + h v_4. \\
\end{align*}
That is,
\begin{align*}
w & = a (\ket{000} + \ket{110}) + b (\ket{001} + \ket{111}) + c (\ket{010} + \ket{100}) + d (\ket{011} + \ket{101}) \\
  & = e (\ket{000} + \ket{011}) + f (\ket{001} + \ket{010}) + g (\ket{100} + \ket{111}) + h (\ket{101} + \ket{110}). \\
\end{align*}
So that:
\begin{align*}
a & = e = d = h = a, \\
b & = f = c = g = b, \\
\end{align*}
and:
\begin{align*}
w & = a \left( \ket{000} + \ket{110} + \ket{011} + \ket{101} \right) + b \left( \ket{001} + \ket{111} + \ket{010} + \ket{100} \right).
\end{align*}
Therefore, the intersection of the subspaces is spanned by:
\[
\ket{000} + \ket{110} + \ket{011} + \ket{101} + \ket{001} + \ket{111} + \ket{010} + \ket{100}
\]
and
\[
\ket{000} + \ket{110} + \ket{011} + \ket{101} - \ket{001} - \ket{111} - \ket{010} - \ket{100},
\]
which are the two codewords of the three qubit phase flip code.

%%%%%%%%%%%%%%%%%%%%%%%%%%%%%%%%%%%%%%%%%%%%%%%%%%%%%%%%%%%%%%%%%%%%%%%%%%%%%%%%%%%%%%%%%%%%%%%%%%%%
\question{10.47}{Verify that the generators of Table \ref{tab:stabilizers} generate the two codewords given by
\begin{align*}
\ket{0} \rightarrow \ket{0_L} & \equiv \frac{(\ket{000} + \ket{111})(\ket{000} + \ket{111})(\ket{000} + \ket{111})}{2\sqrt{2}} \\
\ket{1} \rightarrow \ket{1_L} & \equiv \frac{(\ket{000} - \ket{111})(\ket{000} - \ket{111})(\ket{000} - \ket{111})}{2\sqrt{2}}
\end{align*}

\begin{table}[h!]
\centering
\qtext{
\begin{tabular}{|c|c|}
\hline
Name & Operator \\
\hline
$g_1$          & $Z \; Z \; I \; I \; I \; I \; I \; I \; I$ \\
$g_2$          & $I \; Z \; Z \; I \; I \; I \; I \; I \; I$ \\
$g_3$          & $I \; I \; I \; Z \; Z \; I \; I \; I \; I$ \\
$g_4$          & $I \; I \; I \; I \; Z \; Z \; I \; I \; I$ \\
$g_5$          & $I \; I \; I \; I \; I \; I \; Z \; Z \; I$ \\
$g_6$          & $I \; I \; I \; I \; I \; I \; I \; Z \; Z$ \\
$g_7$          & $X \; X \; X \; X \; X \; X \; I \; I \; I$ \\
$g_8$          & $I \; I \; I \; X \; X \; X \; X \; X \; X$ \\
$\overline{Z}$ & $X \; X \; X \; X \; X \; X \; X \; X \; X$ \\
$\overline{X}$ & $Z \; Z \; Z \; Z \; Z \; Z \; Z \; Z \; Z$ \\
\hline
\end{tabular}
}
\caption{\qtext{Generators for the Shor code, and the logical $Z$ and $X$ operators.}}
\label{tab:stabilizers}
\end{table}

}

%%%%%%%%%%%%%%%%%%%%%%%%%%%%%%%%%%%%%%%%%%%%%%%%%%%%%%%%%%%%%%%%%%%%%%%%%%%%%%%%%%%%%%%%%%%%%%%%%%%%
\question{10.48}{Show that the operations
\begin{align*}
    \overline{Z} & = X_1X_2X_3X_4X_5X_6X_7X_8X_9 \\
    \text{and}\quad \overline{X} & = Z_1Z_2Z_3Z_4Z_5Z_6Z_7Z_8Z_9
\end{align*}
act as logical $Z$ and $X$ operators
on a Shor-coded qubit.
That is, show that this $\overline{Z}$ is independent of and commutes with the generators of the Shor code,
and that $\overline{X}$ is independent of and commutes with the generators of the Shor code,
and anti-commutes with $\overline{Z}$.}


% \printbibliography
% \addcontentsline{toc}{section}{References}

\end{document}
