\documentclass[12pt]{extarticle}
\usepackage{phys460}

\title{PHYS460 - Homework 4}
\author{John Hurst}
\date{October 2024}

\begin{document}
\maketitle

%%%%%%%%%%%%%%%%%%%%%%%%%%%%%%%%%%%%%%%%%%%%%%%%%%%%%%%%%%%%%%%%%%%%%%%%%%%%%%%%%%%%%%%%%%%%%%%%%%%%
\question{A2.13}{Show that every irreducible Abelian matrix group is one-dimensional.
}

%%%%%%%%%%%%%%%%%%%%%%%%%%%%%%%%%%%%%%%%%%%%%%%%%%%%%%%%%%%%%%%%%%%%%%%%%%%%%%%%%%%%%%%%%%%%%%%%%%%%
\question{A2.16}{$S_3$ is the group of permutations of three elements.
Support we order these as mapping 123 to 123; 231; 312; 213; 132, and 321, respectively.
Show that there exist two one-dimensional irreducible representations of $S_3$, one of which is trivial,
and the other of which is 1, 1, 1, -1, -1, -1, corresponding in order to the six permutations given earlier.
Also show that there exists a two-dimensional irreducible representation, with the matrices
\begin{align*}
\begin{pmatrix}
    1 & 0 \\
    0 & 1
\end{pmatrix}, &&
\frac{1}{2}
\begin{pmatrix}
    -1 & -\sqrt{3} \\
    \sqrt{3} & -1
\end{pmatrix}, &&
\frac{1}{2}
\begin{pmatrix}
    -1 & \sqrt{3} \\
    -\sqrt{3} & -1
\end{pmatrix}, \\
\begin{pmatrix}
    -1 & 0 \\
    0 & 1
\end{pmatrix}, &&
\frac{1}{2}
\begin{pmatrix}
    1 & \sqrt{3} \\
    \sqrt{3} & -1
\end{pmatrix}, &&
\frac{1}{2}
\begin{pmatrix}
    1 & -\sqrt{3} \\
    -\sqrt{3} & -1
\end{pmatrix}.
\end{align*}
Verify that the representations are orthogonal.
}

%%%%%%%%%%%%%%%%%%%%%%%%%%%%%%%%%%%%%%%%%%%%%%%%%%%%%%%%%%%%%%%%%%%%%%%%%%%%%%%%%%%%%%%%%%%%%%%%%%%%
\question{A2.24}{Using the results of exercise A2.16, construct the Fourier transform over $S_3$ and express it as a 6x6 unitary matrix.}

%%%%%%%%%%%%%%%%%%%%%%%%%%%%%%%%%%%%%%%%%%%%%%%%%%%%%%%%%%%%%%%%%%%%%%%%%%%%%%%%%%%%%%%%%%%%%%%%%%%%
\question{10.46}{Show that the stabilizer for the three qubit phase flip code is generated by $X_1X_2$ and $X_2X_3$.}

%%%%%%%%%%%%%%%%%%%%%%%%%%%%%%%%%%%%%%%%%%%%%%%%%%%%%%%%%%%%%%%%%%%%%%%%%%%%%%%%%%%%%%%%%%%%%%%%%%%%
\question{10.47}{Verify that the generators of Table \ref{tab:stabilizers} generate the two codewords given by
\begin{align*}
\ket{0} \rightarrow \ket{0_L} & \equiv \frac{(\ket{000} + \ket{111})(\ket{000} + \ket{111})(\ket{000} + \ket{111})}{2\sqrt{2}} \\
\ket{1} \rightarrow \ket{1_L} & \equiv \frac{(\ket{000} - \ket{111})(\ket{000} - \ket{111})(\ket{000} - \ket{111})}{2\sqrt{2}}
\end{align*}

\begin{table}[h!]
\centering
\qtext{
\begin{tabular}{|c|c|}
\hline
Name & Operator \\
\hline
$g_1$          & $Z \; Z \; I \; I \; I \; I \; I \; I \; I$ \\
$g_2$          & $I \; Z \; Z \; I \; I \; I \; I \; I \; I$ \\
$g_3$          & $I \; I \; I \; Z \; Z \; I \; I \; I \; I$ \\
$g_4$          & $I \; I \; I \; I \; Z \; Z \; I \; I \; I$ \\
$g_5$          & $I \; I \; I \; I \; I \; I \; Z \; Z \; I$ \\
$g_6$          & $I \; I \; I \; I \; I \; I \; I \; Z \; Z$ \\
$g_7$          & $X \; X \; X \; X \; X \; X \; I \; I \; I$ \\
$g_8$          & $I \; I \; I \; X \; X \; X \; X \; X \; X$ \\
$\overline{Z}$ & $X \; X \; X \; X \; X \; X \; X \; X \; X$ \\
$\overline{X}$ & $Z \; Z \; Z \; Z \; Z \; Z \; Z \; Z \; Z$ \\
\hline
\end{tabular}
}
\caption{\qtext{Generators for the Shor code, and the logical $Z$ and $X$ operators.}}
\label{tab:stabilizers}
\end{table}

}

%%%%%%%%%%%%%%%%%%%%%%%%%%%%%%%%%%%%%%%%%%%%%%%%%%%%%%%%%%%%%%%%%%%%%%%%%%%%%%%%%%%%%%%%%%%%%%%%%%%%
\question{10.48}{Show that the operations
\begin{align*}
    \overline{Z} & = X_1X_2X_3X_4X_5X_6X_7X_8X_9 \\
    \text{and}\quad \overline{X} & = Z_1Z_2Z_3Z_4Z_5Z_6Z_7Z_8Z_9
\end{align*}
act as logical $Z$ and $X$ operators
on a Shor-coded qubit.
That is, show that this $\overline{Z}$ is independent of and commutes with the generators of the Shor code,
and that $\overline{X}$ is independent of and commutes with the generators of the Shor code,
and anti-commutes with $\overline{Z}$.}


% \printbibliography
% \addcontentsline{toc}{section}{References}

\end{document}
