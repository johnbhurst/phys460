\documentclass[12pt]{extarticle}
\usepackage{phys460}

\title{PHYS460 - Project: HHL Algorithm, Part 2}
\author{John Hurst}
\date{October 2024}

\begin{document}
\maketitle

\tableofcontents

\newpage

%%%%%%%%%%%%%%%%%%%%%%%%%%%%%%%%%%%%%%%%%%%%%%%%%%%%%%%%%%%%%%%%%%%%%%%%%%%%%%%%%%%%%%%%%%%%%%%%%%%%
\section{Introduction}

In my PHYS440 project \cite{github_project_hhl}, I studied the Harrow-Hassidim-Lloyd (HHL) algorithm for solving linear systems.

I extended the walkthrough in \cite{zaman2023step} to a more general 4x4 matrix example, and implemented this in Mathematica and Qiskit.
The main finding was to show in more detail how controlled ancilla rotations are applied when there are more than two clock qubits.

In the PHYS440 project, following the walkthrough, I used the solution eigenvalues in the implementation,
making it somewhat artificial.

In this project, I explore how the algorithm can be implemented without using the solution eigenvalues.
In particular, the ancilla rotation step is implemented without using the solution eigenvalues,
using quantum gates that are not parameterised by the problem input.

\section{Reciprocal}\label{sec:reciprocal}

The Qiskit \texttt{ExactReciprocal} component \cite{ibm_exact_reciprocal} implements a reciprocal using controlled rotation gates.
In this section we explore the implementation in detail.

I used a Jupyter notebook \cite{github_project_exactreciprocal_matrix} to analyse the behaviour of the \texttt{ExactReciprocal} component.

\begin{table}[h!]
\centering
\begin{tabular}{|c|c|c|c|}
\hline
Input $x$     & Input State  & Output State                                              & Output $s/x$  \\
\hline
$0$           &  $\ket{000}$ &  $\ket{000}$                                              & undefined     \\
$\frac{1}{4}$ &  $\ket{001}$ &  $\ket{101}$                                              & $1$           \\
$\frac{1}{2}$ &  $\ket{010}$ & $\frac{\sqrt{3}}{2} \ket{010} + \frac{1}{2} \ket{110}$    & $\frac{1}{2}$ \\
$\frac{3}{4}$ &  $\ket{011}$ & $\frac{2 \sqrt{2}}{3} \ket{011} + \frac{1}{3} \ket{111}$  & $\frac{1}{3}$ \\
(N/A)         &  $\ket{100}$ & $\ket{100}$                                               & \\
(N/A)         &  $\ket{101}$ & $-\ket{001}$                                              & \\
(N/A)         &  $\ket{110}$ & $-\frac{1}{2} \ket{010} + \frac{\sqrt{3}}{2} \ket{110}$   & \\
(N/A)         &  $\ket{111}$ & $-\frac{1}{3} \ket{011} + \frac{2 \sqrt{2}}{3} \ket{111}$ & \\
\hline
\end{tabular}
\caption{ExactReciprocal for 2 state qubits}
\label{tab:exactreciprocal2}
\end{table}

Table \ref{tab:exactreciprocal2} shows the behaviour of the \texttt{ExactReciprocal} component for 2 state qubits.
This behaviour is generated by the following matrix:
\[
E_2 = \begin{pmatrix}
1 &  0 & 0                  & 0                   & 0 & 0 & 0                  & 0 \\
0 &  0 & 0                  & 0                   & 0 & 1 & 0                  & 0 \\
0 &  0 & \frac{\sqrt{3}}{2} & 0                   & 0 & 0 & \frac{1}{2}        & 0 \\
0 &  0 & 0                  & \frac{2\sqrt{2}}{3} & 0 & 0 & 0                  & \frac{1}{3} \\
0 &  0 & 0                  & 0                   & 1 & 0 & 0                  & 0 \\
0 & -1 & 0                  & 0                   & 0 & 0 & 0                  & 0 \\
0 &  0 & -\frac{1}{2}       & 0                   & 0 & 0 & \frac{\sqrt{3}}{2} & 0 \\
0 &  0 & 0                  & -\frac{1}{3}        & 0 & 0 & 0                  & \frac{2\sqrt{2}}{3} \\
\end{pmatrix}
\]

\begin{table}[h!]
\centering
\begin{tabular}{|c|c|c|c|}
\hline
Input $x$     & Input State  & Output State                                               & Output $s/x$  \\
\hline
$0$           & $\ket{0000}$ & $\ket{0000}$                                               & undefined     \\
$\frac{1}{8}$ & $\ket{0001}$ & $ \ket{1001}$                                              & $1$           \\
$\frac{1}{4}$ & $\ket{0010}$ & $\frac{\sqrt{3}}{2} \ket{0010} + \frac{1}{2} \ket{1010}$   & $\frac{1}{2}$ \\
$\frac{3}{8}$ & $\ket{0011}$ & $\frac{2\sqrt{2}}{3} \ket{0011} + \frac{1}{3} \ket{1011}$  & $\frac{1}{3}$ \\
$\frac{1}{2}$ & $\ket{0100}$ & $\frac{\sqrt{15}}{4} \ket{0100} + \frac{1}{4} \ket{1100}$  & $\frac{1}{4}$ \\
$\frac{5}{8}$ & $\ket{0101}$ & $\frac{2\sqrt{6}}{5} \ket{0101} + \frac{1}{5} \ket{1101}$  & $\frac{1}{5}$ \\
$\frac{3}{4}$ & $\ket{0110}$ & $\frac{\sqrt{35}}{6} \ket{0110} + \frac{1}{6} \ket{1110}$  & $\frac{1}{6}$ \\
$\frac{7}{8}$ & $\ket{0111}$ & $\frac{4\sqrt{3}}{7} \ket{0111} + \frac{1}{7} \ket{1111}$  & $\frac{1}{7}$ \\
(N/A)         & $\ket{1000}$ & $\ket{1000}$                                               & \\
(N/A)         & $\ket{1001}$ & $-\ket{0001}$                                              & \\
(N/A)         & $\ket{1010}$ & $-\frac{1}{2} \ket{0010} + \frac{\sqrt{3}}{2} \ket{1010}$  & \\
(N/A)         & $\ket{1011}$ & $-\frac{1}{3} \ket{0011} + \frac{2\sqrt{2}}{3} \ket{1011}$ & \\
(N/A)         & $\ket{1100}$ & $-\frac{1}{4} \ket{0100} + \frac{\sqrt{15}}{4} \ket{1100}$ & \\
(N/A)         & $\ket{1101}$ & $-\frac{1}{5} \ket{0101} + \frac{2\sqrt{6}}{5} \ket{1101}$ & \\
(N/A)         & $\ket{1110}$ & $-\frac{1}{6} \ket{0110} + \frac{\sqrt{35}}{6} \ket{1110}$ & \\
(N/A)         & $\ket{1111}$ & $-\frac{1}{7} \ket{0111} + \frac{4\sqrt{3}}{7} \ket{1111}$ & \\
\hline
\end{tabular}
\caption{ExactReciprocal for 3 state qubits}
\label{tab:exactreciprocal3}
\end{table}

\setcounter{MaxMatrixCols}{20}

This behaviour is generated by the following matrix:
\begin{small}
\[
E_3 = \begin{pmatrix}
1 &  0 & 0                   & 0                   & 0                   & 0                   & 0                   & 0                   & 0 & 0  & 0                  & 0                   & 0                   & 0                   & 0                   & 0 \\
0 &  0 & 0                   & 0                   & 0                   & 0                   & 0                   & 0                   & 0 & 1  & 0                  & 0                   & 0                   & 0                   & 0                   & 0 \\
0 &  0 & \frac{\sqrt{3}}{2}  & 0                   & 0                   & 0                   & 0                   & 0                   & 0 & 0  & \frac{1}{2}        & 0                   & 0                   & 0                   & 0                   & 0 \\
0 &  0 & 0                   & \frac{2\sqrt{2}}{3} & 0                   & 0                   & 0                   & 0                   & 0 & 0  & 0                  & \frac{1}{3}         & 0                   & 0                   & 0                   & 0 \\
0 &  0 & 0                   & 0                   & \frac{\sqrt{15}}{4} & 0                   & 0                   & 0                   & 0 & 0  & 0                  & 0                   & \frac{1}{4}         & 0                   & 0                   & 0 \\
0 &  0 & 0                   & 0                   & 0                   & \frac{2\sqrt{6}}{2} & 0                   & 0                   & 0 & 0  & 0                  & 0                   & 0                   & \frac{1}{5}         & 0                   & 0 \\
0 &  0 & 0                   & 0                   & 0                   & 0                   & \frac{\sqrt{35}}{6} & 0                   & 0 & 0  & 0                  & 0                   & 0                   & 0                   & \frac{1}{6}         & 0 \\
0 &  0 & 0                   & 0                   & 0                   & 0                   & 0                   & \frac{4\sqrt{3}}{7} & 0 & 0  & 0                  & 0                   & 0                   & 0                   & 0                   & \frac{1}{7} \\
0 &  0 & 0                   & 0                   & 0                   & 0                   & 0                   & 0                   & 1 & 0  & 0                  & 0                   & 0                   & 0                   & 0                   & 0 \\
0 &  0 & 0                   & 0                   & 0                   & 0                   & 0                   & 0                   & 0 & -1 & 0                  & 0                   & 0                   & 0                   & 0                   & 0 \\
0 &  0 & -\frac{1}{2}        & 0                   & 0                   & 0                   & 0                   & 0                   & 0 & 0  & \frac{\sqrt{3}}{2} & 0                   & 0                   & 0                   & 0                   & 0 \\
0 &  0 & 0                   & -\frac{1}{3}        & 0                   & 0                   & 0                   & 0                   & 0 & 0  & 0                  & \frac{2\sqrt{2}}{3} & 0                   & 0                   & 0                   & 0 \\
0 &  0 & 0                   & 0                   & -\frac{1}{4}        & 0                   & 0                   & 0                   & 0 & 0  & 0                  & 0                   & \frac{\sqrt{15}}{4} & 0                   & 0                   & 0 \\
0 &  0 & 0                   & 0                   & 0                   & -\frac{1}{5}        & 0                   & 0                   & 0 & 0  & 0                  & 0                   & 0                   & \frac{2\sqrt{6}}{5} & 0                   & 0 \\
0 &  0 & 0                   & 0                   & 0                   & 0                   & -\frac{1}{6}        & 0                   & 0 & 0  & 0                  & 0                   & 0                   & 0                   & \frac{\sqrt{35}}{6} & 0 \\
0 &  0 & 0                   & 0                   & 0                   & 0                   & 0                   & -\frac{1}{7}        & 0 & 0  & 0                  & 0                   & 0                   & 0                   & 0                   & \frac{4\sqrt{3}}{7} \\
\end{pmatrix}
\]
\end{small}

These matrices can be constructed in a general way as follows.
For $n$ state qubits, plus a single ancilla qubit, the matrix $E_n$ is a $2^{n+1} \times 2^{n+1}$ matrix.
The first row would invert 0, but this is meaningless and is not used. The first row has a 1 in the first column, and 0 in the rest of the columns.

For the next $2^n-1$ rows, in the $k$th row (the top half of the matrix), column $2^n+k$ has $\frac{1}{k}$, and column $k$ has $\sqrt{1-\frac{1}{k^2}}$, so that the sum of the amplitudes on the row is 1.
This gives us the functional part of the matrix, i.e. the part that takes $\ket{0}\ket{x}$ to $\sqrt{1-\frac{1}{x^2}}\ket{0}\ket{x} + \frac{1}{x}\ket{1}\ket{x}$.

The elements in the remaining rows are set from their diagonal counterparts to make the entire matrix unitary.

More formally, we can find the matrix $E_n$ as a product of controlled rotations:
\[
E_n = \prod_{k=1}^{2^n-1} E_{n,k}
\]
where

\[
E_{n,k} =
\sum_{b_0\in\{0,1\}} \sum_{b_1\in\{0,1\}} \ldots \sum_{b_n\in\{0,1\}} U_{b_n\ldots b_1b_0} \otimes \outerproduct{b_n} \otimes \ldots \otimes \outerproduct{b_1} \outerproduct{b_0}
\]
and
\[
U_{b_n\ldots b_1b_0} = \begin{cases}
    RY_{\theta_k} & \text{if } b_0b_1\ldots b_n = \text{binary representation of } k \\
    I & \text{otherwise}
\end{cases}
\]
and
\[
\sin(\theta_k) = \frac{1}{k}
\]

\begin{figure}[h]
    \centering
    \begin{tikzpicture}
        \begin{yquant}
            qubit {$q_{\idx} = \ket{b_\idx}$} q[2];
            qubit {$a = \ket{0}$} a;
            [name=rot1] box {$R_Y(\theta_1)$} a | q[0] ~ q[1];
            [name=rot2] box {$R_Y(\theta_2)$} a | q[1] ~ q[0];
            [name=rot3] box {$R_Y(\theta_3)$} a | q[0,1];
        \end{yquant}
        \end{tikzpicture}
    \caption{Reciprocal rotations for 2 state qubits.}
    \label{fig:reciprocalrot2}
\end{figure}

\begin{figure}[h]
    \centering
    \begin{tikzpicture}
        \begin{yquant}
            qubit {$q_{\idx} = \ket{b_\idx}$} q[3];
            qubit {$a = \ket{0}$} a;
            [name=rot1] box {$R_Y(\theta_1)$} a | q[0] ~ q[1,2];
            [name=rot2] box {$R_Y(\theta_2)$} a | q[1] ~ q[0,2];
            [name=rot3] box {$R_Y(\theta_3)$} a | q[0,1] ~ q[2];
            [name=rot4] box {$R_Y(\theta_4)$} a | q[2] ~ q[0,1];
            [name=rot5] box {$R_Y(\theta_5)$} a | q[0,2] ~ q[1];
            [name=rot6] box {$R_Y(\theta_6)$} a | q[1,2] ~ q[0];
            [name=rot7] box {$R_Y(\theta_7)$} a | q[0,1,2];
        \end{yquant}
        \end{tikzpicture}
    \caption{Reciprocal rotations for 3 state qubits.}
    \label{fig:reciprocalrot3}
\end{figure}

\printbibliography
\addcontentsline{toc}{section}{References}



\end{document}
