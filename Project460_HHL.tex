\documentclass[12pt]{extarticle}
\usepackage{phys460}

\title{PHYS460 - Project: HHL Algorithm, Part 2}
\author{John Hurst}
\date{October 2024}

\begin{document}
\maketitle

\tableofcontents

\begin{abstract}
The Harrow-Hassidim-Lloyd (HHL) algorithm\cite{hhl2009} is a well-known quantum algorithm for solving linear systems.

One challenging step in the HHL algorithm is the inversion step, which requires the reciprocal of the eigenvalues of the matrix.

We examine the Qiskit \texttt{ExactReciprocal}\cite{ibm_exact_reciprocal} component in the context of the Harrow-Hassidim-Lloyd (HHL) algorithm,
showing in detail how the component is implemented and how it can be used in the HHL algorithm.
\end{abstract}

\newpage

%%%%%%%%%%%%%%%%%%%%%%%%%%%%%%%%%%%%%%%%%%%%%%%%%%%%%%%%%%%%%%%%%%%%%%%%%%%%%%%%%%%%%%%%%%%%%%%%%%%%
\section{Introduction}

In my PHYS440 project \cite{github_project_hhl}, I studied the Harrow-Hassidim-Lloyd (HHL) algorithm for solving linear systems.

I extended the walkthrough in \cite{zaman2023step} to a more general 4x4 matrix example, and implemented this in Mathematica and Qiskit.
The main finding was to show in more detail how controlled ancilla rotations are applied when there are more than two clock qubits.

In the PHYS440 project, following the walkthrough, I used the solution eigenvalues in the implementation,
making it somewhat artificial.

In this project, I explore how the algorithm can be implemented without using the solution eigenvalues.
In particular, the ancilla rotation step is implemented without using the solution eigenvalues,
using quantum gates that are not parameterised by the problem input.

Figure \ref{fig:hhloverview} shows the overall structure of the HHL algorithm.
\begin{figure}[h]
    \centering
    \begin{tikzpicture}
        \begin{yquant}
            qubit {$a = \ket{0}$} a;
            qubit {$q_{0\ldots n} = \ket{0}$} q[1];
            qubit {$x_{0..\log N} = \ket{b}$} x[1];
            [name=qpe] box {QPE} (q,x);
            [name=inv] box {$\frac{1}{x}$} (a,q);
            [name=iqpe] box {IQPE} (q,x);
            measure a;
            output {$\ket{1}$?} a;
            output {$\ket{\overline{x}}$} x;
        \end{yquant}
        \end{tikzpicture}
    \caption{HHL Overview}
    \label{fig:hhloverview}
\end{figure}

There are three sets of qubits:
\begin{itemize}
    \item $a$ is the ancilla qubit, which is used in controlled rotations for the inversion step, and indicates whether inversion is successful.
    \item $q$ is the ``clock'' register, containing $n$ qubits. which are used to implement the quantum phase estimation (QPE) step.
    \item $x$ is the input/output register. At the beginning, $x$ is prepared with the input state representation of the vector $b$. At the end, if the inversion is successful, $x$ contains the output state representation of estimate $\overline{x}$ of the solution vector $x$.
\end{itemize}

The steps of the algorithm are:
\begin{enumerate}
    \item Prepare the input state $\ket{b}$ in register $x$. This step presents its own practical challenges, which we do not investigate here.
    \item Apply the QPE step to estimate the eigenvalues of the matrix $A$. This step requires the Hamiltionian of the matrix $A$, which again is a practical challenge that we don't address here.
    \item Apply the inversion step to estimate the reciprocal of the eigenvalues.
    \item Apply the inverse QPE step to estimate the solution vector $x$.
    \item Measure the ancilla qubit $a$ to determine if the inversion was successful.
    \item If the inversion was successful, the output register $x$ contains the estimate $\overline{x}$ of the solution vector $x$. We cannot measure the entire value of $\overline{x}$ without losing the benefit of the quantum speedup. Instead we need either to be able to use the quantum state for $\overline{x}$ in the solution of some larger problem, or to be able to measure some aggregate property of $\overline{x}$ that is useful.
\end{enumerate}

Morgan, Ghysels and Mohammadbagherpoor (MGM)\cite{ssrn_enhanced_hhl} study an enhanced version of the HHL algorithm that applies a filtered set of ancilla rotations to improve the efficiency of the algorithm while maintaining the quality of the results.
Jack Morgan's code is available online\cite{github_enhanced_hhl}.
It implements different combinations of strategies for preprocessing and eigenvalue inversion, including "classic" HHL,
the Lee, Joo, Lee (LJL) variant\cite{lee2019hybrid} and the MGM variant.

I studied Jack Morgan's code and adapted simplified versions of it for my own purposes, focusing on the classic eigenvalue inversion step.

\section{Reciprocal}\label{sec:reciprocal}

The Qiskit \texttt{ExactReciprocal} component \cite{ibm_exact_reciprocal} implements a reciprocal using controlled rotation gates.
In this section we explore the implementation in detail.

I used a Jupyter notebook \cite{github_project_exactreciprocal} to analyse the behaviour of the \texttt{ExactReciprocal} component.

The IBM Qiskit documentation for \texttt{ExactReciprocal} states that it maps:
\[
\ket{x}\ket{0} \rightarrow \cos(1/x)\ket{x}\ket{0} + \sin(1/x)\ket{x}\ket{1}
\]
But this is not correct. It actually maps:
\[
\ket{x}\ket{0} \rightarrow \sqrt{1-\frac{1}{x^2}}\ket{x}\ket{0} + \frac{1}{x}\ket{x}\ket{1}
\]
There are also parameters for scaling, and to allow negative values, but we won't examine those in this report.

We will assume a scaling factor of $s=\frac{1}{2^n}$ for $n$ clock qubits, which is suitable when our input is a binary representation of a number between zero and one.
(Jack Morgan's code supports negative eigenvalues and uses an estimate of the maximum eigenvalue to choose the scaling factor.)

So, how does the \texttt{ExactReciprocal} component work?

\begin{table}[h!]
\centering
\begin{tabular}{|c|c|c|c|}
\hline
Input $x$     & Input State  & Output State                                              & Output $s/x$  \\
\hline
$0$           &  $\ket{000}$ &  $\ket{000}$                                              & undefined     \\
$\frac{1}{4}$ &  $\ket{001}$ &  $\ket{101}$                                              & $1$           \\
$\frac{1}{2}$ &  $\ket{010}$ & $\frac{\sqrt{3}}{2} \ket{010} + \frac{1}{2} \ket{110}$    & $\frac{1}{2}$ \\
$\frac{3}{4}$ &  $\ket{011}$ & $\frac{2 \sqrt{2}}{3} \ket{011} + \frac{1}{3} \ket{111}$  & $\frac{1}{3}$ \\
(N/A)         &  $\ket{100}$ & $\ket{100}$                                               & \\
(N/A)         &  $\ket{101}$ & $-\ket{001}$                                              & \\
(N/A)         &  $\ket{110}$ & $-\frac{1}{2} \ket{010} + \frac{\sqrt{3}}{2} \ket{110}$   & \\
(N/A)         &  $\ket{111}$ & $-\frac{1}{3} \ket{011} + \frac{2 \sqrt{2}}{3} \ket{111}$ & \\
\hline
\end{tabular}
\caption{ExactReciprocal for 2 state qubits. ($s=\frac{1}{2^2}=\frac{1}{4}$.)}
\label{tab:exactreciprocal2}
\end{table}

Table \ref{tab:exactreciprocal2} shows the behaviour of the \texttt{ExactReciprocal} component for 2 state qubits.
This behaviour is generated by the following unitary transformation matrix:
\[
E_2 = \begin{pmatrix}
1 &  0 & 0                  & 0                   & 0 & 0 & 0                  & 0 \\
0 &  0 & 0                  & 0                   & 0 & 1 & 0                  & 0 \\
0 &  0 & \frac{\sqrt{3}}{2} & 0                   & 0 & 0 & \frac{1}{2}        & 0 \\
0 &  0 & 0                  & \frac{2\sqrt{2}}{3} & 0 & 0 & 0                  & \frac{1}{3} \\
0 &  0 & 0                  & 0                   & 1 & 0 & 0                  & 0 \\
0 & -1 & 0                  & 0                   & 0 & 0 & 0                  & 0 \\
0 &  0 & -\frac{1}{2}       & 0                   & 0 & 0 & \frac{\sqrt{3}}{2} & 0 \\
0 &  0 & 0                  & -\frac{1}{3}        & 0 & 0 & 0                  & \frac{2\sqrt{2}}{3} \\
\end{pmatrix}
\]

\begin{table}[h!]
\centering
\begin{tabular}{|c|c|c|c|}
\hline
Input $x$     & Input State  & Output State                                               & Output $s/x$  \\
\hline
$0$           & $\ket{0000}$ & $\ket{0000}$                                               & undefined     \\
$\frac{1}{8}$ & $\ket{0001}$ & $ \ket{1001}$                                              & $1$           \\
$\frac{1}{4}$ & $\ket{0010}$ & $\frac{\sqrt{3}}{2} \ket{0010} + \frac{1}{2} \ket{1010}$   & $\frac{1}{2}$ \\
$\frac{3}{8}$ & $\ket{0011}$ & $\frac{2\sqrt{2}}{3} \ket{0011} + \frac{1}{3} \ket{1011}$  & $\frac{1}{3}$ \\
$\frac{1}{2}$ & $\ket{0100}$ & $\frac{\sqrt{15}}{4} \ket{0100} + \frac{1}{4} \ket{1100}$  & $\frac{1}{4}$ \\
$\frac{5}{8}$ & $\ket{0101}$ & $\frac{2\sqrt{6}}{5} \ket{0101} + \frac{1}{5} \ket{1101}$  & $\frac{1}{5}$ \\
$\frac{3}{4}$ & $\ket{0110}$ & $\frac{\sqrt{35}}{6} \ket{0110} + \frac{1}{6} \ket{1110}$  & $\frac{1}{6}$ \\
$\frac{7}{8}$ & $\ket{0111}$ & $\frac{4\sqrt{3}}{7} \ket{0111} + \frac{1}{7} \ket{1111}$  & $\frac{1}{7}$ \\
(N/A)         & $\ket{1000}$ & $\ket{1000}$                                               & \\
(N/A)         & $\ket{1001}$ & $-\ket{0001}$                                              & \\
(N/A)         & $\ket{1010}$ & $-\frac{1}{2} \ket{0010} + \frac{\sqrt{3}}{2} \ket{1010}$  & \\
(N/A)         & $\ket{1011}$ & $-\frac{1}{3} \ket{0011} + \frac{2\sqrt{2}}{3} \ket{1011}$ & \\
(N/A)         & $\ket{1100}$ & $-\frac{1}{4} \ket{0100} + \frac{\sqrt{15}}{4} \ket{1100}$ & \\
(N/A)         & $\ket{1101}$ & $-\frac{1}{5} \ket{0101} + \frac{2\sqrt{6}}{5} \ket{1101}$ & \\
(N/A)         & $\ket{1110}$ & $-\frac{1}{6} \ket{0110} + \frac{\sqrt{35}}{6} \ket{1110}$ & \\
(N/A)         & $\ket{1111}$ & $-\frac{1}{7} \ket{0111} + \frac{4\sqrt{3}}{7} \ket{1111}$ & \\
\hline
\end{tabular}
\caption{ExactReciprocal for 3 state qubits. ($s=\frac{1}{2^3}=\frac{1}{8}$.)}
\label{tab:exactreciprocal3}
\end{table}

Table \ref{tab:exactreciprocal3} shows the behaviour of the \texttt{ExactReciprocal} component for 3 state qubits.
This behaviour is generated by the following unitary transformation matrix:

\setcounter{MaxMatrixCols}{20}
\begin{small}
\[
E_3 = \begin{pmatrix}
1 &  0 & 0                   & 0                   & 0                   & 0                   & 0                   & 0                   & 0 & 0  & 0                  & 0                   & 0                   & 0                   & 0                   & 0 \\
0 &  0 & 0                   & 0                   & 0                   & 0                   & 0                   & 0                   & 0 & 1  & 0                  & 0                   & 0                   & 0                   & 0                   & 0 \\
0 &  0 & \frac{\sqrt{3}}{2}  & 0                   & 0                   & 0                   & 0                   & 0                   & 0 & 0  & \frac{1}{2}        & 0                   & 0                   & 0                   & 0                   & 0 \\
0 &  0 & 0                   & \frac{2\sqrt{2}}{3} & 0                   & 0                   & 0                   & 0                   & 0 & 0  & 0                  & \frac{1}{3}         & 0                   & 0                   & 0                   & 0 \\
0 &  0 & 0                   & 0                   & \frac{\sqrt{15}}{4} & 0                   & 0                   & 0                   & 0 & 0  & 0                  & 0                   & \frac{1}{4}         & 0                   & 0                   & 0 \\
0 &  0 & 0                   & 0                   & 0                   & \frac{2\sqrt{6}}{2} & 0                   & 0                   & 0 & 0  & 0                  & 0                   & 0                   & \frac{1}{5}         & 0                   & 0 \\
0 &  0 & 0                   & 0                   & 0                   & 0                   & \frac{\sqrt{35}}{6} & 0                   & 0 & 0  & 0                  & 0                   & 0                   & 0                   & \frac{1}{6}         & 0 \\
0 &  0 & 0                   & 0                   & 0                   & 0                   & 0                   & \frac{4\sqrt{3}}{7} & 0 & 0  & 0                  & 0                   & 0                   & 0                   & 0                   & \frac{1}{7} \\
0 &  0 & 0                   & 0                   & 0                   & 0                   & 0                   & 0                   & 1 & 0  & 0                  & 0                   & 0                   & 0                   & 0                   & 0 \\
0 &  0 & 0                   & 0                   & 0                   & 0                   & 0                   & 0                   & 0 & -1 & 0                  & 0                   & 0                   & 0                   & 0                   & 0 \\
0 &  0 & -\frac{1}{2}        & 0                   & 0                   & 0                   & 0                   & 0                   & 0 & 0  & \frac{\sqrt{3}}{2} & 0                   & 0                   & 0                   & 0                   & 0 \\
0 &  0 & 0                   & -\frac{1}{3}        & 0                   & 0                   & 0                   & 0                   & 0 & 0  & 0                  & \frac{2\sqrt{2}}{3} & 0                   & 0                   & 0                   & 0 \\
0 &  0 & 0                   & 0                   & -\frac{1}{4}        & 0                   & 0                   & 0                   & 0 & 0  & 0                  & 0                   & \frac{\sqrt{15}}{4} & 0                   & 0                   & 0 \\
0 &  0 & 0                   & 0                   & 0                   & -\frac{1}{5}        & 0                   & 0                   & 0 & 0  & 0                  & 0                   & 0                   & \frac{2\sqrt{6}}{5} & 0                   & 0 \\
0 &  0 & 0                   & 0                   & 0                   & 0                   & -\frac{1}{6}        & 0                   & 0 & 0  & 0                  & 0                   & 0                   & 0                   & \frac{\sqrt{35}}{6} & 0 \\
0 &  0 & 0                   & 0                   & 0                   & 0                   & 0                   & -\frac{1}{7}        & 0 & 0  & 0                  & 0                   & 0                   & 0                   & 0                   & \frac{4\sqrt{3}}{7} \\
\end{pmatrix}
\]
\end{small}

These matrices can be constructed in a general way as follows.
For $n$ state qubits, plus a single ancilla qubit, the \texttt{ExactReciprocal} matrix $E_n$ is a $2^{n+1} \times 2^{n+1}$ matrix.
The first row would invert 0, but this is meaningless and is not used. The first row has a 1 in the first column, and 0 in the rest of the columns.

For the next $2^n-1$ rows, completing the top half of the matrix, in the $k$th row, column $2^n+k$ has $\frac{1}{k}$, and column $k$ has $\sqrt{1-\frac{1}{k^2}}$, so that the sum of the squared amplitudes on the row is 1.
The top half of the matrix is the functional part, i.e. the part that takes $\ket{0}\ket{x}$ to $\sqrt{1-\frac{1}{x^2}}\ket{0}\ket{x} + \frac{1}{x}\ket{1}\ket{x}$.

The bottom half of the matrix specifies the result when the input state of the ancilla qubit is $\ket{1}$, which is not relevant to the function of \texttt{ExactReciprocal}.
The elements in the bottom half are set in such a way to make the entire matrix unitary.

More formally, given $n$ precision or state bits, we can find the \texttt{ExactReciprocal} matrix $E_n$ as a product of $2^{n-1}$ controlled rotations, one for each possible $n$-bit input (except 0):
\[
E_n = \prod_{k=1}^{2^n-1} E_{n,k}
\]
where $E_{n,k}$ is the controlled rotation for input $k$.
Each rotation is controlled by the bit pattern of $k$ using the appropriate tensor product of $\outerproduct{0}$ and $\outerproduct{1}$ projection operators.

\begin{small}
\[
E_{n,k} =
\sum_{b_0\in\{0,1\}} \sum_{b_1\in\{0,1\}} \ldots \sum_{b_{n-1}\in\{0,1\}} U_{b_{n-1}b_{n-2}\ldots b_1b_0} \otimes \outerproduct{b_{n-1}} \otimes \ldots \otimes \outerproduct{b_1} \outerproduct{b_0}
\]
\end{small}

where
\[
U_{b_n\ldots b_1b_0} = \begin{cases}
    R_Y(\theta_k) & \text{if } 0.b_{n-1}b_{n-1}\ldots b_1b_0 = \text{binary representation of } k \\
    I & \text{otherwise}
\end{cases}
\]

Choosing $\theta_k = 2\inv{\sin}(\frac{1}{k})$ such that
\[
\sin\left(\frac{\theta_k}{2}\right) = \frac{1}{k}
\]
gives the desired $\frac{1}{k}$ in the result.

For example, for 2 clock qubits these expand to:
\begin{footnotesize}
\begin{align*}
E_{2,1} &= I \otimes \outerproduct{0} \otimes \outerproduct{0}
         + R_Y(\pi) \otimes \outerproduct{0} \otimes \outerproduct{1}
         + I \otimes \outerproduct{1} \otimes \outerproduct{0}
         + I \otimes \outerproduct{1} \otimes \outerproduct{1} \\
E_{2,2} &= I \otimes \outerproduct{0} \otimes \outerproduct{0}
         + I \otimes \outerproduct{0} \otimes \outerproduct{1}
         + R_Y(\frac{\pi}{3}) \otimes \outerproduct{1} \otimes \outerproduct{0}
         + I \otimes \outerproduct{1} \otimes \outerproduct{1} \\
E_{2,3} &= I \otimes \outerproduct{0} \otimes \outerproduct{0}
            + I \otimes \outerproduct{0} \otimes \outerproduct{1}
            + I \otimes \outerproduct{1} \otimes \outerproduct{0}
            + R_Y\left(2\inv{\sin}\left[\frac{1}{3}\right]\right) \otimes \outerproduct{1} \otimes \outerproduct{1} \\
\end{align*}
\end{footnotesize}
The product, $E_2$, is represented in circuit form in Figure \ref{fig:reciprocalrot2}.

\begin{figure}[h]
    \centering
    \begin{tikzpicture}
        \begin{yquant}
            qubit {$q_{\idx} = \ket{b_\idx}$} q[2];
            qubit {$a = \ket{0}$} a;
            [name=rot1] box {$R_Y(\pi)$} a | q[0] ~ q[1];
            [name=rot2] box {$R_Y(\frac{\pi}{3})$} a | q[1] ~ q[0];
            [name=rot3] box {$R_Y\left(2\inv{\sin}\left[\frac{1}{3}\right]\right)$} a | q[0,1];
        \end{yquant}
        \end{tikzpicture}
    \caption{Reciprocal rotations for 2 state qubits.}
    \label{fig:reciprocalrot2}
\end{figure}

Notice how the controlling bits in $b$ are applied according to the binary representation of the index $k$,
which results in the rotation corresponding to $\frac{1}{k}$ being applied to the ancilla qubit.

% \begin{figure}[h]
%     \centering
%     \begin{tikzpicture}
%         \begin{yquant}
%             qubit {$q_{\idx} = \ket{b_\idx}$} q[3];
%             qubit {$a = \ket{0}$} a;
%             [name=rot1] box {$R_Y(\theta_1)$} a | q[0] ~ q[1,2];
%             [name=rot2] box {$R_Y(\theta_2)$} a | q[1] ~ q[0,2];
%             [name=rot3] box {$R_Y(\theta_3)$} a | q[0,1] ~ q[2];
%             [name=rot4] box {$R_Y(\theta_4)$} a | q[2] ~ q[0,1];
%             [name=rot5] box {$R_Y(\theta_5)$} a | q[0,2] ~ q[1];
%             [name=rot6] box {$R_Y(\theta_6)$} a | q[1,2] ~ q[0];
%             [name=rot7] box {$R_Y(\theta_7)$} a | q[0,1,2];
%         \end{yquant}
%         \end{tikzpicture}
%     \caption{Reciprocal rotations for 3 state qubits.}
%     \label{fig:reciprocalrot3}
% \end{figure}

\section{Conclusion}

The \texttt{ExactReciprocal} component in Qiskit can be thought of as not so much an ``algorithm'' but rather as a lookup table.
For a given number of clock qubits, the component encodes the exact result of the reciprocal for each possible input with that many bits.
Perhaps that is why the component is named \texttt{ExactReciprocal}.

In the formal definition above, the matrix $E_n$ is a product of $2^{n-1}$ sums, with each sum being of $2^n$ terms.
The sums are implicit in the circuit construction of the component, but due to the products, the circuit is still of size $O(2^n)$.
This is in contrast to other quantum algorithms such as QFT, where the circuit size is $O(n^2)$,
and the reason why papers such as \cite{ssrn_enhanced_hhl} focus on reducing the number of ancilla rotations required.

Classical chips often encode numerical functions in lookup tables, which are etched into the hardware.
It is interesting to speculate whether something similar could be done here, if the reciprocal lookup table for a given $n$
could be encoded into a quantum chip, and used as a building block for quantum algorithms.

\printbibliography
\addcontentsline{toc}{section}{References}



\end{document}
